\section{Zielsetzung und Ergebnisse}

Das Ziel dieser Arbeit soll es sein, einen Prototypen zu entwickeln, mit welchem es möglich ist, Hautkrebs-Diagnosen auf einem Smartphone durchzuführen. Die Klassifizierung erfolgt durch ein neuronales Netz und die Applikation soll browserbasiert sein, damit sie von einer großen Zahl an Smartphone-Nutzern bedient werden kann. Weiter soll die Software soweit ausgebaut sein, damit eine Integration in den Arbeitsablauf eines Dermatologen möglich ist.
Um das System einer möglichst großen Anzahl an Personen zugänglich zu machen, wird der Source-Code zudem öffentlich einsehbar sein und ebenso soll die Applikation auf frei verfügbaren Standards wie FHIR aufsetzen. \citet*{FHIR}
Da es sich außerdem um ein medizinisches Produkt handelt, müssen zudem einige weitere Anforderungen erfüllt werden. Eine gute Orientierung sollte hier die ISO 13485 bieten. \citet*{ISO_9001}

\textbf{Abgeleitete Forschungsfrage:}
Wie könnte ein System für die Diagnose von Hautkrebs mit neuronalen Netzten auf Smartphones aussehen und wie könnte eine solche Applikation in den Arbeitsablauf von Dermatologen integriert werden?