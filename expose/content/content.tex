\section{Ausgangssituation und Problemstellung}

Beschreiben Sie hier die Problemstellung der Arbeit, die Sie adressieren möchten. In dieser Sektion kann auch auf bereits existierende Literatur zu diesem Thema \citet[3-4]{Shearer2000} oder auf praktische Lösungsversuche näher eingegangen werden.

\section{Relevanz}

Hier erklären Sie, warum ihre Arbeit theoretisch/praktisch relevant ist. Die LeserIn soll verstehen, worin der Mehrwert dieser Arbeit liegt. Üblicherweise hat eine Arbeit theoretische Relevanz, weil sie eine Wissenslücke im wissenschaftlichen Bereich schließt, oder praktische Relevanz, weil sie ein praktisches Problem adressiert.

\section{Zielsetzung und Ergebnisse}

Welches wissenschaftliche/praktische Ziel verfolgen Sie mit Ihrer Arbeit und welche Ergebnisse (z.B. Prototyp, Handlungsempfehlungen, usw.) erwarten Sie? Versuchen Sie hier so präzise wie möglich zu sein. Je genauer die Zielsetzung formuliert ist, desto einfach ist es in der Regel die Arbeit durchzuführen.

\textbf{Abgeleitete Forschungsfrage:} Welche Relevanz besitzen mehrschichtige Schätzmodelle, gegenüber konventionellen Schätzmodellen bei der Vorhersage von Zeitreihen?


\section{Vorgehensweise und Methoden}

Hier beschreiben Sie, wie sie das gesetzte Ziel erreichen wollen. Achten Sie dabei auf drei Dinge:

\begin{itemize}
	\item Welches theoretische Rahmenwerk (z.B. Design Science) – wenn es so etwas in Ihrer Arbeit gibt - kommt in der Arbeit zum Einsatz.
	\item Wie sieht der Ablauf der Forschungsarbeit aus und welche Schritte müssen ausgeführt werden.
	\item Welche Methoden kommen zum Einsatz um diese Schritte auszuführen.
\end{itemize}
