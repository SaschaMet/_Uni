
\section{Vorgehensweise und Methoden}

Als Methoden werde ich den ASUM-DM sowie den Design Science Ansatz verwenden.
ASUM-DM stellt eine Schritt-für-Schritt-Anleitung zur Verfügung und umfasst fünf Phasen:

\begin{enumerate}
	\item Analyse
	\item Design
	\item Entwicklung
	\item Inbetriebnahme
	\item Betrieb und Optimierung
\end{enumerate}

Diese Anleitung stellt ein gutes Grundgerüst dar, auf welcher die Applikation entwickelt und betrieben werden kann. \cite*{asum_dm} Ziel des Design Science Ansatzes ist das Erlangen von Wissen und von Verständnis eines Problembereichs durch den Aufbau und die Anwendung eines Artefakts. Er liefert nicht nur die Anforderungen an die Forschung, sondern definiert auch Akzeptanzkriterien für die abschließende Bewertung der Forschungsergebnisse. \cite*{design_science}

Beide Ansätze lassen sich gut miteinander verbinden und ermöglichen eine ganzheitliche Betrachtung der Problemstellung. Mein Vorgehen zur Beantwortung der Forschungsfrage wird wie folgt aussehen:

\begin{enumerate}
	\item Analyse des Ist-Zustandes\\
	      Hier werde ich den aktuellen Prozess bis zu einer Hautkrebs-Diagnose beschreiben, aktuelle durch Computer gestützte Diagnosemöglichkeiten beleuchten, bestehende Anbieter und Modelle vorstellen und auf die Durchführbarkeit von Hautkrebs-Diagnosen auf Smartphones eingehen.
	\item Theoretische Problembehandlung\\
	      In diesem Abschnitt werde ich untersuchen, welche Herausforderungen bei der Umsetzung dieses Projektes existieren.
	      Im Detail werde ich mich mit folgenden Fragen befassen:
	      \begin{itemize}
		      \item Welche Vor- und Nachteile hat die Telemedizin?
		      \item Welche Einsatzmöglichkeiten bestehen für ein solches System?
		      \item Welche Anforderungen muss eine solche Applikation erfüllen?
	      \end{itemize}
	      Zudem möchte ich Dermatologen befragen, um ein möglichst detailliertes Bild über deren Herausforderungen und mögliche Lösungsansätze zu bekommen, sowie später evaluieren zu können inwieweit die entwickelte Applikation ihren Zweck erfüllt.
	\item Praktische Problembehandlung\\
	      Dieser Teil wird sich mit der Erstellung der gesamten Applikation beschäftigen, ebenso wie mit der Integration in den Alltag von Dermatologen. Außerdem werde ich hier das Convolutional Neural Network entwickeln und evaluieren, welches die Diagnose übernehmen soll.
\end{enumerate}