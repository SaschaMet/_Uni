\section{Relevanz}

Mit Convolutional Neural Networks ist es mittlerweile möglich, Melanome mit einer hohen Präzision zu diagnostizieren. Moderne Netze übertreffen sogar die Leistung von Hautärzten bei der Klassifizierung von Hautkrebs.
Mit solchen Netzen ausgestattet können mobile Endgeräte die Reichweite von Dermatologen außerhalb ihrer Kliniken und Arztpraxen potenziell erweitern. Es wird prognostiziert, dass es bis zum Jahr 2021 über 6 Milliarden Smartphone Nutzer geben wird. Damit würde für viele Menschen ein kostengünstiger und universeller Zugang zu einer diagnostischen Versorgung bestehen. \citet*{Stanford_CNN}

Zudem ist der Ausbau der Telemedizin in den letzten Jahren immer stärker in den Fokus gerückt. Von 2019 zu 2020 ist die Nachfrage nach einer Videosprechstunde um 9 Prozentpunkte gestiegen. \citet*{demand_telemedicine}
Zusätzlich hat die Corona-Pandemie dafür gesorgt, dass vor allem während der Spitze der Pandemie im Frühjar 2020 die Arztbesuche unter anderem bei Onkologen um bis zu 50 Prozent zurückgegangen sind. Dies führt dazu, dass weniger Menschen ihre Vorsorgeuntersuchungen wahrnehmen, obwohl diese vor allem bei Hautkrebs wichtig für eine frühzeitige Diagnose sind. \citet*{corona_medical_specialist}